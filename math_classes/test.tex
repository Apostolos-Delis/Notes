\documentclass[a4paper,11pt,autostyle=true]{article}
\usepackage[letterpaper, portrait, margin=0.5in, amsmath]{geometry}


% define the title
\author{H.~Partl}
\title{\LaTeX{} Test}
\begin{document}
% generates the title
\maketitle
\begin{abstract}
The abstract abstract.
This is a Test.
\end{abstract}
% insert the table of contents
\tableofcontents
\section{Some Interesting Words}
Well, and here begins my lovely article.
\section{Good Bye World}
\ldots{} and here it ends.
% Example 1
\ldots when Einstein introduced his formula
\begin{equation}
e = m \cdot c^2 \; ,
\end{equation}
which is at the same time the most widely known
and the least well understood physical formula.

% Example 2
\ldots from which follows Kirchhoff's current law:
\begin{equation}
\sum_{k=1}^{n} I_k = 0 \; .
\end{equation}
Kirchhoff's voltage law can be derived \ldots
% Example 3
\ldots which has several advantages.
\begin{equation}
I_D = I_F - I_R
\end{equation}
is the core of a very different transistor model. \ldots
``Please press the `x' key.''
\begin{equation}
    J_a(b) = a + b / 32.b
\end{equation}
    
Test\footnote{Very Long Test text sdfsdaf s dfsf  fds  fsd f s fds f sd f sd % 
    fdsfsdafsdaf sda fsda f sd  fasd fsdf adsf adfs }


This is text style:
$\lim_{n \to \infty}
 \sum_{k=1}^n \frac{1}{k^2}
 = \frac{\pi^2}{6}$.
And this is display style:
 \begin{equation}
  \lim_{n \to \infty}
  \sum_{k=1}^n \frac{1}{k^2}
  = \frac{\pi^2}{6}
 \end{equation}

 This is a test for all:
 \begin{equation}
 $\forall x \in \mathbf{R}:
 \qquad x^{2} \geq 0$
 \end{equation}
 
 Math 115AH Notation:
 \begin{equation}
     \phi_\beta(x) = [x]_\beta
 \end{equation}
 
 Other tests:
 \begin{equation}
 \sqrt{x} \Leftrightarrow x^{1/2}
 \quad \sqrt[3]{2}
 \quad \sqrt{x^{2} + \sqrt{y}}
 \quad \surd[x^2 + y^2]$
 \end{equation}
 
Derivatives:
\begin{equation*}
  \sqrt{\frac{x^2}{k+1}}\qquad
  x^\frac{2}{k+1}\qquad
  \frac{\partial^2f}
  {\partial x^2}
\end{equation*}

Isomorphisms
\begin{equation*}
 f_n(x) \stackrel{*}{\approx} 1
\end{equation*}

Big Operators
(These should be on a new line)
\begin{equation}
\sum_{i=1}^n \qquad
\int_0^{\frac{\pi}{2}} \qquad
\prod_\epsilon
\end{equation}

Subset stuff
\begin{equation}
\sum^n_{\substack{0<i<n \\
        j\subseteq i}}
   P(i,j) = Q(i,j)
\end{equation}

Matrix stuff 
\begin{equation*}
    \mathbf{X} = \left(
      \begin{array}{ccc}
        x_1 & x_2 & \ldots \\
        x_3 & x_4 & \ldots \\
        \vdots & \vdots & \ddots
      \end{array} \right)
  \end{equation*}

Piecewise Functions
\begin{equation}
  |x| = \left\{
    \begin{array}{rl}
      -x & \text{if } x < 0,\\
      0 & \text{if } x = 0,\\
      x & \text{if } x > 0.
    \end{array} \right.
\end{equation}

Working with new commands:
\newcommand{\ud}{\,\mathrm{d}}
\begin{equation}
 \int_a^b f(x)\ud x
\end{equation}

Different Integral Spacing:

\begin{IEEEeqnarray}{c}
  \int\int f(x)g(y)
\ud x \ud y \\
  \int\!\!\!\int
         f(x)g(y) \ud x \ud y \\
  \iint f(x)g(y)  \ud x \ud y
\end{IEEEeqnarray}

Alignment stuff:
\begin{equation*}
{}^{14}_{6}\text{C}
\qquad \text{versus} \qquad
{}^{14}_{\phantom{1}6}\text{C}
\end{equation*}

Table of Real Numbers\footnote{The last two require amssymb or amsfonts}:
\Re \qquad
  \mathcal{R} \qquad
  \mathfrak{R} \qquad
  \mathbb{R} \qquad $

Complex Equations:
\begin{equation}
 P = \frac{\displaystyle{
   \sum_{i=1}^n (x_i- x)
   (y_i- y)}}
   {\displaystyle{\left[
   \sum_{i=1}^n(x_i-x)^2
   \sum_{i=1}^n(y_i- y)^2
   \right]^{1/2}}}
\end{equation}

Theorems:

\begin{name}[some text]
This is my interesting theorem \end{name}


\newtheorem{mur}{Murphy}[section]
\begin{mur} If there are two or
more ways to do something, and
one of those ways can result in
a catastrophe, then someone
will do it.\end{mur}

\begin{proof}
    Proof: Trivial, use
 \begin{equation*}
   E=mc^2.
 \end{equation*}
\end{proof}

\begin{proof}
  This is a proof that ends
  with a numbered equation:
  \begin{equation}
    a = b + c. \qedhere
  \end{equation}
\end{proof}

Math 115AH test:
Let T:V\rightarrow{} V, x \mapsto{} 2x, Then:
%be an endomorphism, then

\begin{equation}
    \sum_{i=1}^n \lambda_i \dot T(x_i) \in V
\end{equation}

Partl~\cite{pa} has
proposed that \ldots
\begin{thebibliography}{99}
\bibitem{pa} H.~Partl:
\emph{German \TeX},
TUGboat Volume~9, Issue~1 (1988)
\end{thebibliography}

Graphs:
\newline
\setlength{\unitlength}{5cm}
\begin{picture}(1,1)
  \put(0,0){\line(0,1){1}}
  \put(0,0){\line(1,0){1}}
  \put(0,0){\line(1,1){1}}
  \put(0,0){\line(1,2){.5}}
  \put(0,0){\line(1,3){.3333}}
  \put(0,0){\line(1,4){.25}}
  \put(0,0){\line(1,5){.2}}
  \put(0,0){\line(1,6){.1667}}
  \put(0,0){\line(2,1){1}}
  \put(0,0){\line(2,3){.6667}}
  \put(0,0){\line(2,5){.4}}
  \put(0,0){\line(3,1){1}}
  \put(0,0){\line(3,2){1}}
  \put(0,0){\line(3,4){.75}}
  \put(0,0){\line(3,5){.6}}
  \put(0,0){\line(4,1){1}}
  \put(0,0){\line(4,3){1}}
  \put(0,0){\line(4,5){.8}}
  \put(0,0){\line(5,1){1}}
  \put(0,0){\line(5,2){1}}
  \put(0,0){\line(5,3){1}}
  \put(0,0){\line(5,4){1}}
  \put(0,0){\line(5,6){.8333}}
  \put(0,0){\line(6,1){1}}
  \put(0,0){\line(6,5){1}}
\end{picture}

Arrows:
\newline
\setlength{\unitlength}{0.75mm}
\begin{picture}(60,40)
  \put(30,20){\vector(1,0){30}}
  \put(30,20){\vector(4,1){20}}
  \put(30,20){\vector(3,1){25}}
  \put(30,20){\vector(2,1){30}}
  \put(30,20){\vector(1,2){10}}
  \thicklines
  \put(30,20){\vector(-4,1){30}}
  \put(30,20){\vector(-1,4){5}}
  \thinlines
  \put(30,20){\vector(-1,-1){5}}
  \put(30,20){\vector(-1,-4){5}}
\end{picture}

Misc figures:
\newline
\setlength{\unitlength}{0.8cm}
\begin{picture}(6,5)
  \thicklines
  \put(1,0.5){\line(2,1){3}}
  \put(4,2){\line(-2,1){2}}
  \put(2,3){\line(-2,-5){1}}
  \put(0.7,0.3){$A$}
  \put(4.05,1.9){$B$}
  \put(1.7,2.95){$C$}
  \put(3.1,2.5){$a$}
  \put(1.3,1.7){$b$}
  \put(2.5,1.05){$c$}
  \put(0.3,4){$F=
    \sqrt{s(s-a)(s-b)(s-c)}$}
  \put(3.5,0.4){$\displaystyle
    s:=\frac{a+b+c}{2}$}
\end{picture}

Grid:
\newline
\setlength{\unitlength}{2mm}
\begin{picture}(30,20)
  \linethickness{0.075mm}
  \multiput(0,0)(1,0){26}%
    {\line(0,1){20}}
  \multiput(0,0)(0,1){21}%
    {\line(1,0){25}}
  \linethickness{0.15mm}
  \multiput(0,0)(5,0){6}%
    {\line(0,1){20}}
  \multiput(0,0)(0,5){5}%
    {\line(1,0){25}}
  \linethickness{0.3mm}
  \multiput(5,0)(10,0){2}%
    {\line(0,1){20}}
  \multiput(0,5)(0,10){2}%
    {\line(1,0){25}}
\end{picture}

\newline
FileSystem: 
\newline
\setlength{\unitlength}{0.5mm}
\begin{picture}(120,168)
\newsavebox{\foldera}
\savebox{\foldera}
  (40,32)[bl]{% definition
  \multiput(0,0)(0,28){2}
    {\line(1,0){40}}
  \multiput(0,0)(40,0){2}
    {\line(0,1){28}}
  \put(1,28){\oval(2,2)[tl]}
  \put(1,29){\line(1,0){5}}
  \put(9,29){\oval(6,6)[tl]}
  \put(9,32){\line(1,0){8}}
  \put(17,29){\oval(6,6)[tr]}
  \put(20,29){\line(1,0){19}}
  \put(39,28){\oval(2,2)[tr]}
}
\newsavebox{\folderb}
\savebox{\folderb}
  (40,32)[l]{%         definition
  \put(0,14){\line(1,0){8}}
  \put(8,0){\usebox{\foldera}}
}
\put(34,26){\line(0,1){102}}
\put(14,128){\usebox{\foldera}}
\multiput(34,86)(0,-37){3}
  {\usebox{\folderb}}
\end{picture}

Special Relativity:
\setlength{\unitlength}{0.8cm}
\begin{picture}(6,4)(-3,-2)
  \put(-2.5,0){\vector(1,0){5}}
  \put(2.7,-0.1){$\chi$}
  \put(0,-1.5){\vector(0,1){3}}
  \multiput(-2.5,1)(0.4,0){13}
    {\line(1,0){0.2}}
  \multiput(-2.5,-1)(0.4,0){13}
    {\line(1,0){0.2}}
  \put(0.2,1.4)
    {$\beta=v/c=\tanh\chi$}
  \qbezier(0,0)(0.8853,0.8853)
    (2,0.9640)
  \qbezier(0,0)(-0.8853,-0.8853)
    (-2,-0.9640)
  \put(-3,-2){\circle*{0.2}}
\end{picture}

\newline
More Text stuff:
{\small The small and
\textbf{bold} Romans ruled}
{\Large all of great big
\textit{Italy}.}


\makebox[\textwidth]{%
    c e n t r a l}\par
\makebox[\textwidth][s]{%
    s p r e a d}\par
\framebox[1.1\width]{Guess I'm
    framed now!} \par
\framebox[0.8\width][r]{Bummer,
    I am too wide} \par
\framebox[1cm][l]{never
    mind, so am I}
Can you read this?


\raisebox{0pt}[0pt][0pt]{\Large%
\textbf{Aaaa\raisebox{-0.3ex}{a}%
\raisebox{-0.7ex}{aa}%
\raisebox{-1.2ex}{r}%
\raisebox{-2.2ex}{g}%
\raisebox{-4.5ex}{h}}}
she shouted, but not even the next
one in line noticed that something
terrible had happened to her.

\end{document}

