\documentclass{article}
\usepackage[utf8]{inputenc}
\usepackage[english]{babel}
\usepackage{amsmath}
\usepackage{amssymb}
\usepackage{amsthm}
 
\newtheorem{theorem}{Theorem}[section]
\newtheorem{corollary}{Corollary}[theorem]
\newtheorem{lemma}[theorem]{Lemma}
\newtheorem{definition}[theorem]{Definition}
\newtheorem{proposition}[theorem]{Proposition}

\title{Math 131B - Real Analysis and Metric-Space Topology}
 
\begin{document}
\section{Math 131B - Real Analysis and Metric-Space Topology}
Apostolos Delis, UCLA 2021

\begin{definition}
A metric space is a pair $(X, d)$ where $X$ is a nonempty set and
$d: X \times X \rightarrow [0, \infty)$
\begin{enumerate}
    \item $d(x,y) = 0\Leftrightarrow x = y$
    \item $d(x,y) = d(y,x)$
    \item $d(x,z) \le d(x,y) + d(y,z)$
\end{enumerate}
\end{definition}

\begin{definition}
    Let $(x_n)_{n=1}^{\infty}$ a sequence in $X$, then $\lim_{n \to \infty} d(x_n, x) = 0$
\end{definition}

\begin{proposition}
    Let $(x_n)_{n=1}^{\infty}$ be a sequence in a metrix space, let $x_0 \in X$ and
    $r>0$, then the open ball centered at $x_0$ of radius $r$ is
    $B(x_0,r)=\{x\in X: d(x,x_0) < r\}$
\end{proposition}

\begin{definition}
    Let $(X, d)$ be a metrix space. Let $E \subseteq X, x_0 \in X$ We say that
    \begin{itemize}
        \item $x_0$ is an interior point of $E$ if $\exists r > 0$ such that
            $B(x_0,r)\subseteq E$
        \item $x_0$ is an exterior point of $E$ if $x_0$ is an interior point of
            $X\setminus E$
        \item $x_0$ is a boundary point if it is neither an interior or an exterior point
        \item We say that $E$ is open iff $\forall x \in E, \exists r > 0$ such that
            $B(x_0,r) \subseteq E$
        \item $E$ is closed if $X\setminus E$ is open
    \end{itemize}
\end{definition}

\begin{definition}
    Let $(X, d)$ be a metrix space, $E \subseteq X$, then
    \begin{equation}
        int(E) = \{x_0 \in X: \text{x is interior to }E\}
    \end{equation}
    \begin{equation}
        ext(E) = \{x_0 \in X: \text{x is exterior to }E \}
    \end{equation}
    \begin{equation}
        \partial(E) = \{x_0 \in X: \text{x is a boundary point of }E\}
    \end{equation}
\end{definition}

\begin{proposition}
    Let $(X, d)$ be a metrix space, $x, y, z \in X$. Then 
    \begin{equation}
        |d(x,y) - d(x,z)| \leq d(y,z)
    \end{equation}
    Furthermore, $\forall w \in X$,
    \begin{equation}
        |d(x,y) - d(w,z)| \leq d(x,w) + d(y,z)
    \end{equation}
\end{proposition}

\begin{proof}
    $|d(x,y) - d(x,z)| = \max\{d(x,y) - d(x,z), d(x,z) - d(x,y)\}$ so it suffices to show
    that $d(x,y) - d(x,z) \leq d(y,z)$ and $d(x,z) - d(x,y) \leq d(y,z)$ or \\
    $d(x,y) \leq d(y,z) + d(x,z)$ and $d(x,z) \leq d(y,z) + d(x,z)$ \\
    These are instances of the triangle inequality, \\
    $|d(x,y) - d(x,z) + d(x,z) -d(w,z)| \leq |d(x,y) - d(x,z)| + | d(x,z) - d(w,z)|$
    $\leq |d(y,z) + d(x,w)|$
\end{proof}

\begin{proposition}
    Let $x_1, x_2, ... , x_n \in X$. Then let $(X, d)$ be a metric space. Then
    $d(x_1, x_n) \leq d(x_1, x_2) + ... + d(x_{n-1}, x_n)$
\end{proposition}

\begin{proof}
    (Proof by Induction) \\
    Base case: let $n = 3$, then this follows the triangle inequality \checkmark\\
    For $n > 3$, we have that (by the induction hypothesis)\\
    $d(x_1, x_{n+1})\leq d(x_1, x_n)+d(x_n, x_{n+1})\leq d(x_1, x_2)+...+d(x_n, x_{n+1})$
\end{proof}

\begin{proposition}
    Let $x_1, x_2, ... , x_n \in X$, then the sequence $(x_n)_{n=1}^{\infty}$ converges
    to x iff any subsequence $(x_{n_k})_{k=1}^{\infty})$ converges to $x$
\end{proposition}

\begin{proof}
    $(\Rightarrow)$ If $x_n \rightarrow x$, then $d(x_n, x) \rightarrow 0$, so
    $d(x_{n_k}, x) \rightarrow 0$ for any subsequence $(x_{n_k})_{k=1}^{\infty})$ \\
    $(\Leftarrow)$ let $a_n = d(x_n, x)$. Suppose $x_n \rightarrow x$, then 
    $a_n \nrightarrow 0 \Rightarrow \exists (n_k) \forall x_{n_{k_j})} \nrightarrow x$,
    thus $\exists\; \epsilon > 0\; \forall N \; \exists\; n \geq N$ 
    such that $a_n \geq \epsilon$
\end{proof}

\begin{definition}
    A set $E$ is open iff $\forall x \in E$, $x\in int(E)$
    \begin{itemize}
        \item E is open iff $E = int(E)$
        \item E is open $\Leftrightarrow \partial E \cap E  = \emptyset$
    \end{itemize}
\end{definition}

\begin{proposition}
    Every open ball $B(x_0, r_)$ is open
\end{proposition}
\begin{proof}
    Let $x\in B(x_0, r)$, then $d(x_0, x) < r$. Now let $r^{\prime} = r -d(x_0, x) > 0$,
    then the claim is that $B(x, r^{\prime}) \subseteq B(x_0, r)$. To prove this,
    let $y \in B(x, r^{\prime})$, then $d(x_0, y) \leq d(x_0, x) + d(x, y)$ by the
    triange inequality, and since $r^{\prime} = r - d(x_0, x)$, and
    $d(x,y) < r^{\prime}$, we have $d(x_0, y) < d(x_0, x) + r^{\prime} = r$
\end{proof}

\begin{proposition}
    If $x_0 \in X$, then $\{x_0\}$ is closed.
\end{proposition}
\begin{proof}
    Want to show $X\setminus\{x_0\}$ is open. \\
    Let $x \in X\setminus\{x_0\}$ and $r = d(x_0,x)$, now want to show that
    $B(x,r) \subseteq X\setminus\{x_0\}$. It suffices to show that
    $x_0 \not\in B(x,r)$, which is evident since $d(x,x_0) = r \not< r$
\end{proof}


\end{document}
