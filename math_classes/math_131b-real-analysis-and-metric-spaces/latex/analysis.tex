\documentclass{article}
\usepackage[utf8]{inputenc}
\usepackage[english]{babel}
\usepackage{amsmath}
\usepackage{amssymb}
\usepackage{amsthm}

\newtheorem{theorem}{Theorem}[section]
\newtheorem{corollary}{Corollary}[theorem]
\newtheorem{lemma}[theorem]{Lemma}
\newtheorem{definition}[theorem]{Definition}
\newtheorem{proposition}[theorem]{Proposition}
\newtheorem{remark}[theorem]{Remark}

\title{Math 131B - Real Analysis and Metric-Space Topology}

\begin{document}
\section{Math 131B - Real Analysis and Metric-Space Topology}
Apostolos Delis, UCLA 2021

\begin{definition}
A metric space is a pair $(X, d)$ where $X$ is a nonempty set and
$d: X \times X \rightarrow [0, \infty)$
\begin{enumerate}
    \item $d(x,y) = 0\Leftrightarrow x = y$
    \item $d(x,y) = d(y,x)$
    \item $d(x,z) \le d(x,y) + d(y,z)$
\end{enumerate}
\end{definition}

\begin{definition}
    Let $(x_n)_{n=1}^{\infty}$ a sequence in $X$, then $\lim_{n \to \infty} d(x_n, x) = 0$
\end{definition}

\begin{proposition}
    Let $(x_n)_{n=1}^{\infty}$ be a sequence in a metrix space, let $x_0 \in X$ and
    $r>0$, then the open ball centered at $x_0$ of radius $r$ is
    $B(x_0,r)=\{x\in X: d(x,x_0) < r\}$
\end{proposition}

\begin{definition}
    Let $(X, d)$ be a metrix space. Let $E \subseteq X, x_0 \in X$ We say that
    \begin{itemize}
        \item $x_0$ is an interior point of $E$ if $\exists r > 0$ such that
            $B(x_0,r)\subseteq E$
        \item $x_0$ is an exterior point of $E$ if $x_0$ is an interior point of
            $X\setminus E$
        \item $x_0$ is a boundary point if it is neither an interior or an exterior point
        \item We say that $E$ is open iff $\forall x \in E, \exists r > 0$ such that
            $B(x_0,r) \subseteq E$
        \item $E$ is closed if $X\setminus E$ is open
    \end{itemize}
\end{definition}

\begin{definition}
    Let $(X, d)$ be a metrix space, $E \subseteq X$, then
    \begin{equation}
        int(E) = \{x_0 \in X: \text{x is interior to }E\}
    \end{equation}
    \begin{equation}
        ext(E) = \{x_0 \in X: \text{x is exterior to }E \}
    \end{equation}
    \begin{equation}
        \partial(E) = \{x_0 \in X: \text{x is a boundary point of }E\}
    \end{equation}
\end{definition}

\begin{proposition}
    Let $(X, d)$ be a metrix space, $x, y, z \in X$. Then
    \begin{equation}
        |d(x,y) - d(x,z)| \leq d(y,z)
    \end{equation}
    Furthermore, $\forall w \in X$,
    \begin{equation}
        |d(x,y) - d(w,z)| \leq d(x,w) + d(y,z)
    \end{equation}
\end{proposition}

\begin{proof}
    $|d(x,y) - d(x,z)| = \max\{d(x,y) - d(x,z), d(x,z) - d(x,y)\}$ so it suffices to show
    that $d(x,y) - d(x,z) \leq d(y,z)$ and $d(x,z) - d(x,y) \leq d(y,z)$ or \\
    $d(x,y) \leq d(y,z) + d(x,z)$ and $d(x,z) \leq d(y,z) + d(x,z)$ \\
    These are instances of the triangle inequality, \\
    $|d(x,y) - d(x,z) + d(x,z) -d(w,z)| \leq |d(x,y) - d(x,z)| + | d(x,z) - d(w,z)|$
    $\leq |d(y,z) + d(x,w)|$
\end{proof}

\begin{proposition}
    Let $x_1, x_2, ... , x_n \in X$. Then let $(X, d)$ be a metric space. Then
    $d(x_1, x_n) \leq d(x_1, x_2) + ... + d(x_{n-1}, x_n)$
\end{proposition}

\begin{proof}
    (Proof by Induction) \\
    Base case: let $n = 3$, then this follows the triangle inequality \checkmark\\
    For $n > 3$, we have that (by the induction hypothesis)\\
    $d(x_1, x_{n+1})\leq d(x_1, x_n)+d(x_n, x_{n+1})\leq d(x_1, x_2)+...+d(x_n, x_{n+1})$
\end{proof}

\begin{proposition}
    Let $x_1, x_2, ... , x_n \in X$, then the sequence $(x_n)_{n=1}^{\infty}$ converges
    to x iff any subsequence $(x_{n_k})_{k=1}^{\infty})$ converges to $x$
\end{proposition}

\begin{proof}
    $(\Rightarrow)$ If $x_n \rightarrow x$, then $d(x_n, x) \rightarrow 0$, so
    $d(x_{n_k}, x) \rightarrow 0$ for any subsequence $(x_{n_k})_{k=1}^{\infty})$ \\
    $(\Leftarrow)$ let $a_n = d(x_n, x)$. Suppose $x_n \rightarrow x$, then
    $a_n \nrightarrow 0 \Rightarrow \exists (n_k) \forall x_{n_{k_j})} \nrightarrow x$,
    thus $\exists\; \epsilon > 0\; \forall N \; \exists\; n \geq N$
    such that $a_n \geq \epsilon$
\end{proof}

\begin{definition}
    A set $E$ is open iff $\forall x \in E$, $x\in int(E)$
    \begin{itemize}
        \item E is open iff $E = int(E)$
        \item E is open $\Leftrightarrow \partial E \cap E  = \emptyset$
    \end{itemize}
\end{definition}

\begin{proposition}
    Every open ball $B(x_0, r_)$ is open
\end{proposition}
\begin{proof}
    Let $x\in B(x_0, r)$, then $d(x_0, x) < r$. Now let $r^{\prime} = r -d(x_0, x) > 0$,
    then the claim is that $B(x, r^{\prime}) \subseteq B(x_0, r)$. To prove this,
    let $y \in B(x, r^{\prime})$, then $d(x_0, y) \leq d(x_0, x) + d(x, y)$ by the
    triange inequality, and since $r^{\prime} = r - d(x_0, x)$, and
    $d(x,y) < r^{\prime}$, we have $d(x_0, y) < d(x_0, x) + r^{\prime} = r$
\end{proof}

\begin{proposition}
    If $x_0 \in X$, then $\{x_0\}$ is closed.
\end{proposition}
\begin{proof}
    Want to show $X\setminus\{x_0\}$ is open. \\
    Let $x \in X\setminus\{x_0\}$ and $r = d(x_0,x)$, now want to show that
    $B(x,r) \subseteq X\setminus\{x_0\}$. It suffices to show that
    $x_0 \not\in B(x,r)$, which is evident since $d(x,x_0) = r \not< r$
\end{proof}

\begin{proposition}
    Let $(X,d)$ be a metric space. Then
    \begin{enumerate}
        \item $\emptyset$ and $X$ are both open and closed.
        \item If $\{A_i : i \in I\}$ is a collection of open sets, then
            $\bigcup\limits_{i \in I} A_i$ is open
        \item If $A_1, ... A_n$ is a finite group of sets, then
            $\bigcap\limits_{i \in I} A_1$ is open
    \end{enumerate}
\end{proposition}
\begin{proof}
    \begin{enumerate}
        \item $\emptyset$ contains no points so it is vacuosly open. $X$ contains all
            balls so clearly it is open. Then \\
            $X \setminus \emptyset = X$ so $\emptyset$ is closed
            $X \setminus X = \emptyset$ so $X$ is closed
        \item Let $x \in \bigcup\limits_{i\in I} A_i$ then $\forall i \in I$ such that
            $x \in A_i$. As $A_i$ is open, $\exists r > 0$ such that
            $B(x,r) \subseteq A_i$, as $A_i \subseteq \bigcup\limits_{i\in I} A_i$, then
            $B(x,r) \subseteq \bigcup\limits_{i\in I} A_i$ so $A_i$ is open
        \item Let $x\in \bigcap\limits_{i\in I} A_i$, then $\forall i \in \{1, ..., n\}$,
            $x \in A_i$, as every $A_i$ is open, $\exists r_i > 0$ such that
            $B(x,r_i) \subseteq A_i$. So let $r = \min\{r_1, ... , r_n\} > 0$, then as
            $r \leq r_i$, $B(x,r) \subseteq B(x,r_i) \subseteq A_i$ so
            $B(x,r) \subseteq \bigcap\limits_{i\in I} A_i$ so
            $\bigcap\limits_{i\in I} A_i$ is open
    \end{enumerate}
\end{proof}

\begin{corollary}
    \begin{enumerate}
        \item If $\{F_i : i \in I\}$ are closed sets, then
            $\bigcap\limits_{i\in I} F_i$ is closed
        \item If $F_1, ..., F_n$ are closed, then
            $\bigcup\limits_{i\in I} F_i$ is closed
    \end{enumerate}
\end{corollary}
\begin{proof}
    \begin{enumerate}
        \item $X \setminus \bigcap\limits_{i\in I} F_i$ =
            $\bigcup\limits_{i\in I} (X\setminus F_i)$
        \item $X \setminus \bigcup\limits_{i\in I} F_i$ =
            $\bigcap\limits_{i\in I} (X\setminus F_i)$
    \end{enumerate}
\end{proof}

\begin{proposition}
    If $(X,d)$ is a metric space, and $E \subseteq X$, then $E$ is open iff it is a union
    of open balls
\end{proposition}
\begin{proof}
    ($\Leftarrow$) As open balls are open, unions of open sets are open \\
    ($\Rightarrow$) if $x \in E$, then $\exists r_x>0$, such that $B(x,r_x)\subseteq E$,
    then $\bigcap\limits_{x\in E} B(x,r_x) \subseteq E$ as each $B(x,r_x) \subseteq E$,
    but if $x_0 \in E$, then
    $x_0 \in B(x_0, r_{x_0}) \subseteq \bigcup\limits_{x\in E} B(x_0, r_x)$, so
    $E \subseteq \bigcap\limits_{x\in E} B(x_0, r_x)$ so
    $E = \bigcap\limits_{x\in E} B(x_0, r_x)$
\end{proof}

\begin{definition}
    Let $(X, d)$ be a metric space, and $E \subseteq X$. We say $x_0 \in X$ is an
    adherent point of $E$ if $\forall r > 0$, $B(x_0, r) \cap E \not= \emptyset$
\end{definition}

\begin{proposition}
    Let $(X, d)$ be a metric space. $E \subseteq X, x \in X$ then the following are
    equivalent:
    \begin{enumerate}
        \item $x$ is an adherent point of $E$
        \item $x$ is an interior point of $E$
        \item $\exists$ sequence $(x_n)_{n=1}^{\infty}$ such that $x_n \in E$
            $forall n \geq 1$ and $x = \lim_{n\rightarrow \infty} x_n$
    \end{enumerate}
\end{proposition}
\begin{proof}
    $(1) \Leftrightarrow (2)$ is an exercise, note: $(2)$ is equivalent to $x$ is
    \textbf{not} an exterior point \\
    $(1) \Rightarrow (3)$: $\forall n \geq 1$, $B(x, \frac{1}{n}) \cap E \not= \emptyset$.
    So let $x_n \in B(x, \frac{1}{n})$. Then $(x_n)_{n=1}^{\infty}$ is a sequence such
    that $x_n \in E \forall n \geq 1$. \\
    Then $d(x, x_n) < \frac{1}{n}$ so
    $\lim_{n\rightarrow \infty} d(x_n, x) = 0$ \\
    $(3) \Rightarrow (1)$: Let $x \in X$, let $(x_n)_{n=1}^{\infty}$, be a sequence such
    that $x_n \in E \forall n \geq 1$ and $x_n \rightarrow x$ as $n \rightarrow \infty$.
    Given $r > 0$, $\exists N \geq 1$ such that $\forall n \geq N, d(x_n, x) < r$, so
    $x_N \in B(x,r) \cap E \not= \emptyset$
\end{proof}

\begin{definition}
    If $(X,d)$ is a metric space and $E\subseteq X$, the closure of $E$ is the set:
    \begin{equation}
        \overline{E} = \{x \in X : \text{x is an adherent point}\}
    \end{equation}
\end{definition}
\begin{remark}
    From the proposition above, we have that
    \begin{equation}
        \overline{E} = int(E) \cup \partial E
    \end{equation}
\end{remark}

\begin{proposition}
    The closure $\overline{E}$ of a set $E$ is closed
\end{proposition}
\begin{proof}
    Prove $X \setminus \overline{E}$ is open. Suppose that $X \setminus E$ is closed.
    Then $\exists x \in X \setminus \overline{E}$ and $\exists r > 0$ so that
    $B(x,r) \cap E \not= \emptyset$. So $\forall r >  0$ let
    $y \in B(x, \frac{r}{2}) \cap \overline{E}$. As $y \in \overline{E}$,
    $B(y, \frac{r}{2}) \cap \overline{E} \not= \emptyset$ so
    $\exists z \in B(y, \frac{r}{2}) \cap E$, then
    $d(x,z) \leq d(x,y) + d(y,z) < \frac{r}{2} + \frac{r}{2} = r$ so
    $z \in B(x, r) \cap E$
\end{proof}

\begin{proposition}
    Let $(X,d)$ be a metric space, $E \subseteq X$, then the followig are equal
    \begin{enumerate}
        \item $E$ is closed
        \item $E = \overline{E}$
        \item If $(x_n)_{n=1}^{\infty}$ is a convergent subsequence and $x_n \in E$
            $\forall n \geq 1$, then $\lim_{n \to \infty} x_n \in E$
    \end{enumerate}
\end{proposition}
\begin{proof}
    $(1) \Leftrightarrow (2)$: $E$ is closed $\Leftrightarrow X \setminus E$ is open
    $\Leftrightarrow \forall x \in X \setminus E, \exists r > 0$ such that
    $B(x,r) \subseteq X \setminus E \Leftrightarrow x \not\in \overline{E}$
    $\Leftrightarrow x \in X \setminus \overline{E}$ so
    $X \setminus E = X \setminus \overline{E}$ so $E = \overline{E}$ \\
    $(2) \Rightarrow (3)$: As $E = \overline{E}$, use the result from earlier
\end{proof}

\begin{definition}
    Let $(X, d)$ a metric space, $Y \subseteq X$. Let $d_y$ be the restriction of d to
    $Y \times Y$, then $(Y, d_y)$ is an induced metric space from $(X, d)$. A subset
    $E\subseteq Y$ is relatively open if $E$ is open in $(Y, d_y)$
\end{definition}
\begin{theorem}
    Let $(X, d)$ a metric space, $Y \subseteq X, E \subseteq Y$
    \begin{enumerate}
        \item E is relatively open $\Leftrightarrow \exists A \subseteq X$ open in
            $(X,d)$ such that $E = A \cap Y$
        \item E is relatively closed $\Leftrightarrow \exists B \subseteq X$ close in
            $(X,d)$ such that $E = B \cap Y$
    \end{enumerate}
\end{theorem}
\begin{proof}
    Denote $B_{X}(x_0, r) = \{x \in X : d(x, x_0) < r\}$,
    $B_{Y}(y_0, r) = \{y \in Y : d(x_0, y) < r)\}$. Observe that
    $B_{Y}(y_0, r) = B_{X}(x_0, r) \cap Y$ \\
    $(\Rightarrow)$: $E$ is relatively open $\Rightarrow\forall x\in E,\exists r_x > 0$
    such that $B_{Y}(y_0, r) \subseteq E$, $(E = \bigcap\limits_{x\in E} B_{X}(x, r_x))$
    put $A = \bigcap\limits_{x\in E} B_{X}(x, r_x)$, observe that $A$ is open in $X$, and
    that $A \cap Y = (\bigcap\limits_{x\in E} B_{X}(x, r_x)) \cap Y =$
    $\bigcap\limits_{x\in E} (B_X(x, r) \cap Y)= \bigcap\limits_{x\in E} B_{Y}(x, r_x)=E$
    \\
    $(\Leftarrow)$ Let $x \in E$. Need $r_x > 0$ such that $B_{Y}(x, r_x) \subseteq E$.
    Note $x \in E \subseteq A$. Since $A \subseteq X$ is open, $\exists r_x > 0$ such
    that $B_{X}(x, r_x) \subseteq A \Rightarrow B_{Y}(x, r_x) = B_{X}(x, r_x) \cap Y$
    $\subseteq A \cap Y = E \Rightarrow E$ is relatively open
\end{proof}

\begin{definition}
    Let $(x_n)_{n=1}^{\infty}$ be a sequence in $(X, d)$. We say $(x_n)_{n=1}^{\infty}$
    is Cauchy if $\forall \epsilon > 0$, $\exists N \in \mathbb{N}, \forall m, n > N$,
    $d(x_n, x_m) < \epsilon$
\end{definition}

\begin{lemma}
    If $(x_n)_{n=1}^{\infty})$ converges, then it is Cauchy
    Note: Converse is not true
\end{lemma}
\begin{proof}
    Say $\lim_{n\to \infty} x_n = x$. Then $\forall \epsilon > 0,\exists N\in \mathbb{N}$
    $\forall n \geq N$, $d(x, x_n) < \epsilon, \Rightarrow \forall n,m \geq N$,
    $d(x_n, x_m) \leq d(x_n, x) + d(x_m, x) < 2\epsilon$
\end{proof}

\end{document}
