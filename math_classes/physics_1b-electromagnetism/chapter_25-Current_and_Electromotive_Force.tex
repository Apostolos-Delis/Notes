\documentclass[11pt, a4paper]{article}
\usepackage[letterpaper, portrait, margin=0.5in]{geometry}
\usepackage[english]{babel}  % force American English hyphenation patterns
\usepackage{amsmath,mathtools}

\usepackage{graphicx}
\usepackage{wrapfig}


\begin{document}
\title{Chapter 25 Current, Resistance, and Electromotive Force}
\author{Apostolos Delis}
\date{\today}
\maketitle

\tableofcontents
\section[25.1 Current]{Current}
\begin{itemize}
    \item A current is any motion of charge from one region to another
    \item In electrostatic situations, the electric field is zero everywhere within the
        conductor, and there is no current. However, this does not mean that all charges
        within the conductor are at rest.
    \item Consider what happens if a constant, steady electric field $\vec{\mathbf{E}}$
        is established inside a conductor. A particle inside the conducting material is
        then subjectued to steady force $\vec{\mathbf{F}} = q \vec{\mathbf{E}}$
    \item If the charged particle were moving in a vacuum, this steady force would cause
        a steady acceleration in the direction of $\vec{\mathbf{F}}$, and after a time,
        the charged particle would be moving in that direction at high speed
    \item A charged particle moving through a conductor however undergoes frequent
        collisions with massive, nearly stationary ions of the material. Such collisions
        cause the particle's direction of motion to undergo random change, with a net
        effect of the electric field $\vec{\mathbf{E}}$ is that three is a very slow net
        motion or drift of the moving charged particles as a group in the direction of
        the electric force $\vec{\mathbf{F}} = q \vec{\mathbf{E}}$
\end{itemize}

\subsection{The Direction of Current Flow}
\begin{itemize}
    \item The electric field $\vec{\mathbf{E}}$ does work on the moving charges,
        resulting in kinetic energy transferrred to the material of the conductor by
        means of collisions with the ions
    \item A conventional current is treated as a flow of positive charges, regardless of
        whether the free charges in the conductor are positive, negative, or both.
    \item In a metallic conductor, the moving charges are electrons, but the current
        still points in the direction positive charges would flow.
    \item If the net charge $dQ$ flows through an area  in a time $dt$, the current $I$
        through the area is:
        \begin{equation}
            I = \frac{dQ}{dt}
        \end{equation}
\end{itemize}

\subsection{Current, Drift Velocity, and Current}
\begin{itemize}
    \item We can express current in terms of the drift velocity of the moving charges.
    \item Suppose there are $n$ moving charged particles per unit volume, then $n$ is the
        concentration of particles, now suppose they all have a drift velocity with
        magnitude $v_d$
    \item In a time interval $dt$, each particle moves a distance $v_d dt$
    \item The volume of the cylinder is $A v_d dt$ and the number of particles is
        $nAv_{d}dt$. If each particle has a charge $q$, the charge $dQ$ flows out of the
        end of the cylinder during the time $dt$ is:
        \begin{equation}
            dQ = q(nA v_d dt) = nq v_d A dt
        \end{equation}
    \item The current is then defined as:
        \begin{equation}
            I = \frac{dQ}{dt} = nqv_d A
        \end{equation}
    \item The current per unit cross-sectional area is called the \textbf{current
            density} $J$
        \begin{equation}
            J = \frac{I}{A} = nqv_d
        \end{equation}
    \item So to summarize, $I$ is the current through an area, $\frac{dQ}{dt}$ is the
        rage at which charge flows though area, $n$ is concentration of moving charged
        particles, $q$ is charge per particle, $v_d$ is drift speed, and $A$ is
        cross-sectional area
\end{itemize}

\section[25.2, Resistivity]{Resistivity}
\begin{itemize}
    \item The current density $\vec{\mathbf{J}}$ in a conductor depends on the electric
        field $\vec{\mathbf{E}}$ and on the properties of the material
    \item For some materials, especially metals, at a given temperature,
        $\vec{\mathbf{J}}$ is n:nohearly directly proportional to $\vec{\mathbf{E}}$, and the
        ratio of the magnitudes of $E$ and $J$ is nearly constant. This relationship is
        known as Ohm's Law
    \item We define the \textbf{resistivity} $\rho$ in terms of the magnitude of the
        electric field divided by the magnitude of the current density caused by that
        field:
        \begin{equation}
            \rho = \frac{E}{J}
        \end{equation}
    \item The reciprocal of resistivity is \textbf{conductivity}. With units
        $(\Omega\cdot m)^{-1}$
    \item A material that obey's Ohm's law reasonably well is called an ohmic conductor
        or a linear conductor, for such materials, $\rho$ is a constant that does not
        depend on the value of $E$
\end{itemize}

\subsection{Resistivity and Temperature}
\begin{itemize}
    \item The resistivity of a metallic conductor nearly always increases with increasing
        temperature. As temperature increases, the ions in the conductor vibrate with
        greater amplitude, making it more likely that a moving electron will collide with
        an ion
    \item Over a small temperature range (up to $100 C^{\circ}$), the resistivity of a
        metal can be represented approximately by the equation:
        \begin{equation}
            \rho(T) = \rho_0 [1 + \alpha(T-T_0)]
        \end{equation}
        where $\rho_0$ is the resistivity at a reference temperature $T_0$ and $\alpha$
        is the temperature coefficient of resistivity
\end{itemize}

\section[25.3 Resistance]{Resistance}
\begin{itemize}
    \item For a conductor of resistivity $\rho$, the current density $\vec{\mathbf{J}}$
        at a point where the electric field is $\vec{\mathbf{E}}$ is given by:
        \begin{equation}
            \vec{\mathbf{E}} = \rho \vec{\mathbf{J}}
        \end{equation}
    \item Suppose our conductor is a wire with unifrom cross-sectional area $A$ and
        length $L$. Let $V$ be the potential difference between the higher-potential and
        lower-potential ends of the conductor, so that $V$ is positive
    \item The direction of the current is always from the higher-potential end to the
        lower-potential end, and this is because the current in a conductor flows in the
        direction of $\vec{\mathbf{E}}$ and $\vec{\mathbf{E}}$ points in the direction of
        decreasing electric potential
    \item We can also relate the value of the current $I$ to the potential difference
        between the ends of the conductor. If the magnitudes of the current density
        $\vec{\mathbf{J}}$ and the electric field $\vec{\mathbf{E}}$ are unform
        throughout the conductor, the total current $I = JA$ and the potential difference
        is $V = EL$, which leads to:
        \begin{equation}
            \frac{V}{L} = \frac{\rho I}{A} \; \; \; \text{or} \; \; \; V =
            \frac{\rho L}{A} I
        \end{equation}
    \item The ratio of $V$ to $I$ for a particular conductor is called its
        \textbf{resistance} $R$
        \begin{equation}
            R = \frac{V}{I} = \frac{\rho L}{A}
        \end{equation}
        where $\rho$ is the resistivity of the conductor material, $L$ is the length of
        the conductor, and $A$ is the cross-sectional area
    \item If $\rho$ is constant, then so is $R$, this leads to the equation often called
        Ohm's Law:
        \begin{equation}
            V = IR
        \end{equation}
        Where $V$ is voltage between ends of a conductor, $I$ is the current in the
        conductor, and $R$ is the resistance in the conductor
    \item Note that because resistivity of ta material varies with temperature, the
        resistance of a specific conductor also varies with temperature. For small
        temperature ranges (similar to $\rho$), we have that:
        \begin{equation}
            R(T) = R_0 [1 + \alpha(T - T_0)]
        \end{equation}
\end{itemize}

\subsection{Electromotive Force}
\begin{itemize}
    \item \textbf{Electromotive force} is the influence that makes current flow from
        lower to higher potential
    \item Every complete circuit with a steady current must include a source of emf such
        as batteries, electric generators, solar cells, etc
    \item The SI unit of emf is the same as that for potential, the volt ($1V = 1J / C$).
        A typical flashlight battery has an emf of $1.5V$, so the battery does $1.5J$ of
        work on every coulomb of charge that passes through it.
    \item An example of an ideal source of emf that maintains the potential differenve
        between the conductors $a$ and $b$, called terminals, has associated with this
        potential difference an electric field $\vec{\mathbf{E}}$ in the region around
        the terminals
    \item A charge $q$ within the source experiences an electric force
        $\vec{\mathbf{F}}_e = q \vec{\mathbf{E}}$, but the source also provides an
        additional influence, which is the nonelectrostatic force $\vec{\mathbf{F}}_n$,
        which pushes charge from $b$ to $a$ in an uphill direction against the electric
        force $\vec{\mathbf{F}}_e$
    \item If a positive charge $q$ is moved from $b$ to $a$ inside the source, the
        nonelectrostatic force $\vec{\mathbf{F}}_n$ does a positive amount of work
        $W_n = q \mathcal{E}$ on the charge, and is opposite to the electrostatic force
        $\vec{\mathbf{F}}_e$, so the potential energy increases by an $qV_{ab}$
    \item By combining previous equations, we can see that the potential difference
        between the ends of the wire is given by:
        \begin{equation}
            \mathcal{E} = V_{ab} = IR
        \end{equation}
\end{itemize}

\subsection{Internal Resistance}
\begin{itemize}
    \item Real sources of emf in a circuit don't behave in exactly the way described
        since the potential difference across a real source in a circuit is not equal to
        the emf, since the charge moving through the material of a real source
        encounters resistance.
    \item This \textbf{Internal Resistance} $r$ of the source begaves according to Ohm's
        Law, $r$ is constant and independent of the current $I$.
    \item As the current moves through $r$, it experiences an associated drop in
        potential equal to $Ir$, so the potential difference $V_{ab}$ between the
        terminals is:
        \begin{equation}
            V_{ab} = \mathbf{E} - Ir
        \end{equation}
    \item The potential $V_{ab}$ is called \textbf{terminal voltage} and is less than the
        emf $\mathcal{E}$ because of the term $Ir$ representing the potential drop
        across the internal resistance $r$
    \item The increase in potential energy $qV_{ab}$ as a charge $q$ moves from $b$ to
        $a$ within the source is less than the work $q\mathcal{E}$ done by the
        nonelectrostatic force $\vec{\mathbf{F}}_n$, since some potential energy is lost
        in traversing the internal resistance
\end{itemize}

\section[25.5, Energy and Power In Electric Circuits0]{Energy and Power in Electric Circuits}
\begin{itemize}
    \item As the amount of charge $q$ passes through the circuit element, there is a
        change in potential energy equal to $qV_{ab}$
    \item If the potential at $a$ is lower than at $b$, then $V_{ab}$ is negative and
        there is a net transfer of energy out the circuit, so $qV_{ab}$ can denote the
        quantity of energy that is either delivered to a circuit element or exracted from
        that element
    \item If the current though the element is $I$, then in a time interval $dt$, an
        amount of charge $dQ = Idt$ passes through the element. The potential energy
        change for this amount of charge is $V_{ab}dQ = V_{ab}Idt$
    \item Dividing this expression by $dt$, we obtain the rate at which energy is
        transferrred, also known as power, denoted as:
        \begin{equation}
            P = V_{ab}I
        \end{equation}
    \item If the circuit element is a resistor, the potential difference is $V_{ab} = IR$
        The power is thus:
        \begin{equation}
            P = V_{ab}I = I^2 R = \frac{V_{ab}^{2}}{R}
        \end{equation}
        where $R$ is the resistance of the resistor and $I$ is the current in the
        resistor
    \item For the power output of a source, we have that if $a$ is a higher potential
        than point $b$, then $V_{a} > V_{b}$ and $V_{ab}$ is positive, then energy is
        being delivered to the external circuit at a rate of:
        \begin{equation}
            P = V_{ab}I
        \end{equation}
        where
        \begin{equation}
            V_{ab} = \mathcal{E} - Ir
        \end{equation}
        so
        \begin{equation}
            P = V_{ab}I = \mathcal{E}I - I^2 r
        \end{equation}
    \item For power input into source, we have that the curent $I$ in the circuit is
        opposite to what it was for power output, so we then have that:
        \begin{equation}
            V_{ab} = \mathcal{E} + Ir
        \end{equation}
        and than we see that for power:
        \begin{equation}
            P = V_{ab}I = \mathcal{E}I + I^2 r
        \end{equation}
\end{itemize}

\end{document}
