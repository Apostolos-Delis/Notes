\documentclass[11pt, a4paper]{article}
\usepackage[letterpaper, portrait, margin=0.5in]{geometry}
\usepackage[english]{babel}  % force American English hyphenation patterns
\usepackage{amsmath,mathtools}

\usepackage{graphicx}
\usepackage{wrapfig}


\begin{document}
\title{Chapter Capacitance and Dielectrics}
\author{Apostolos Delis}
\date{\today}
\maketitle

\tableofcontents
\section[24.1, Capacitors and Capacitance]{Capacitors and Capacitance}
\begin{itemize}
    \item A capacitor is a device that stores electric potential energy and electric charge
    \item Any two conductors seperated by an insulator form a capacitor, in most
        applications, capacitors start with zero net charge and electrons are transferred
        from one conductor to another
    \item The two conductors have charges with equal magnitude and opposite sign, and the
        net charge as a whole remains zero
    \item When a capacitor is said to have a charge $Q$, it means that one end will have
        a charge of $+Q$ and one side will have a charge $-Q$
    \item The electric field at any point in the region between the conductors is
        proportional to the magnitude $Q$ of the charge on each conductor. It follows
        that the potential difference $V_{ab}$ between conductors is also proportional to
        $Q$
    \item The ration of the charge to potential difference is known as
        \textbf{capacitance} $C$ of the capacitor:
        \begin{equation}
            C = \frac{Q}{V_{ab}}
        \end{equation}
\end{itemize}

\subsection{Calculating Capacitance: Capacitors in Vacuum}
\begin{itemize}
    \item We can calculate the capacitance $C$ of a given capacitor by finding the
        potential difference $V_{ab}$ between conductors for a given charge $Q$, for now
        consider the capacitors in a vaccum, so the empty space that seperates the
        conductors of the capacitor
    \item The simplest capacitor is two parallel conducting plates with area $A$, called
        a \textbf{parallel-plate capacitor}
    \item By using Gauss's law, we found that $E = \sigma / \epsilon_0$, where $\sigma$
        is the magnitude of the surface charge density of each plate. so alternatively,
        we have that $\sigma = Q / A$, so the field magnitude can be expressed as:
        \begin{equation}
            E = \frac{\sigma}{\epsilon_0} = \frac{Q}{\epsilon_0 A}
        \end{equation}
    \item The field is uniform and the distance between the plates is $d$, so the
        potential difference between the two plates is:
        \begin{equation}
            V_{ab} = Ed = \frac{1}{\epsilon_0} \frac{Qd}{A}
        \end{equation}
    \item For any capacitor in vaccum, the capacitance $C$ only depends on the shapes,
        dimensions, and seperation of conductors that make up the capacitor
\end{itemize}

\section[24.2, Capacitors in Series and Parallel]{Capacitors in Series and Parallel}
Capacitors can be combined in many ways to achieve the results needed, the simplest ways
to combine capacitors is to put them in series or in parallel.

\subsection{Capacitors in Series}
\begin{itemize}
    \item Two capacitors are connected in series by conducting wires between points $a$
        and $b$. When a constant positive potential difference $V_{ab}$ is applied
        between points $a$ and $b$, the capacitors become charged
    \item When you have $n$ capacitors, the equivalent capacitance can be found from the
        reciprocals of the sums of all the capacitors
        \begin{equation}
            \frac{1}{C_{eq}} = \frac{1}{C_1} + \frac{1}{C_2} + ... + \frac{1}{C_n}
        \end{equation}
\end{itemize}

\subsection{Capacitors in Parallel}
\begin{itemize}
    \item When capacitors are connected in parallel, there can be much higher total
        capacitance, since the total amount of capacitor area is increased
    \item When you have $n$ capacitors, the equivalent capacitance is a sum of all the
        individual capacitors' capacitance:
        \begin{equation}
            C_{eq} = C_1 + C_2 + C_3 + ... + C_n
        \end{equation}
\end{itemize}

\section{24.3, Energy Storage in Capacitors and Electric-Field Energy}
\begin{itemize}
    \item We can calculate the potential energy $U$ of a charged capacitor by calculating
        the work $W$ required to charge it. When a capacitor is charged, the final charge
        is $Q$ and the potential difference is $V$
        \begin{equation}
            V = \frac{Q}{C}
        \end{equation}
    \item Let $v$ and $q$ be the charge and potential difference, then $q = q / C$, then
        the work $dW$ required to transfer an additional element of charge $dq$ is
        \begin{equation}
            dW = vdq = \frac{qdq}{C}
        \end{equation}
    \item The total work $W$ to increase the capacitor charge $q$ from zero to $Q$ is
        \begin{equation}
            {W} = \int_0^{W} dW = \frac{1}{C}\int_0^{Q} qdq = \frac{Q^2}{2C}
        \end{equation}
    \item We can define the potential energy of an uncharged capacitor to be zero, then
        $W$ is equal to the potential energy $U$ of the charged capacitor. The final
        charge is $Q = CV$ so $U$ can be expressed as:
        \begin{equation}
            U = \frac{Q^2}{2C} = \frac{1}{2}CV^2 = \frac{1}{2}QV
        \end{equation}
\end{itemize}
\end{document}
