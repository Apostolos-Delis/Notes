\documentclass[11pt, a4paper]{article}
\usepackage[letterpaper, portrait, margin=0.5in]{geometry}
\usepackage[english]{babel}  % force American English hyphenation patterns
\usepackage{amsmath,mathtools}

\usepackage{graphicx}
\usepackage{wrapfig}


\begin{document}
\title{Chapter 23 Electric Potential}
\author{Apostolos Delis}
\date{\today}
\maketitle

\tableofcontents
\section[23.1, Electric Potential Energy]{Electric Potential Energy}
\begin{itemize}
    \item When a force $\vec{\mathbf{F}}$ acts on a particle that moves from point $a$ to
        point $b$, the work done $W_{a\rightarrow b}$ by the force is given by the line
        integral:
        \begin{equation}
            W_{a\rightarrow b} = \int_a^b \vec{\mathbf{F}} \cdot d \vec{\mathbf{l}} =
            \int_a^b F\cos\phi dl
        \end{equation}
        where $d \vec{\mathbf{l}}$ is an infinitesimal displacement along the particle's
        path and $\phi$ is the angle between $\vec{\mathbf{F}}$ and $d \vec{\mathbf{l}}$
        at each point along the path
    \item If the force $\vec{\mathbf{F}}$ is conservative, the work done by
        $\vec{\mathbf{F}}$ can always be expressed in terms of \textbf{potential energy}
        $U$. Then we have that
        \begin{equation}
            W_{a\rightarrow b} = U_a - U_b = -(U_a - U_b) = -\Delta U
        \end{equation}
    \item The work energy theorem states that the change in kinetic energy $\Delta K$
        during the displacement equals the total work done on the particle. Then total
        mechanical energy is conserved with $K_a + U_a = K_b + U_b$
\end{itemize}

\subsection{Electric Potential Energy in a Uniform Field}
\begin{itemize}
    \item Suppose a pair of parallel metal plates setup a uniform downward electric field
        with magnitude $E$. The field exerts a downward force with magnitude $F = q_0E$
        on a positive test charge $q_0$.
    \item The charge moves down a distance $d$, from point $a$ to $b$, so the total work
        done can be calculated as follows:
        \begin{equation}
            W_{a\rightarrow b} = Fd = q_0 Ed
        \end{equation}
    \item Whether the test charge is positive or negative, when it moves with the field
        and decreases when it moves against the field. $U$ increases if the test charge
        $q_0$ moves in the direction opposite of the electric force
        $\vec{\mathbf{F}}=q_0 \vec{\mathbf{E}}$, but increases when $q_0$ moves in the
        same direction as the electric force $\vec{\mathbf{F}}$
\end{itemize}

\subsection{Electric Potential Energy of Two Point Charges}
\begin{itemize}
    \item It is usefull to calculate the work done on a test charge $q_0$ moving in the
        electric field caused by a single, stationary point charge $q$
    \item Consider the radial component along the radial line, the force on $q_0$ is
        given by Coulomb's law,
        \begin{equation}
            F_r = \frac{1}{4\pi\epsilon_0}\frac{qq_0}{r^2}
        \end{equation}
    \item The force is not constant during the displacement, so we must integrate to
        calculate $W_{a\rightarrow b}$ done on $q_0$ by this force:
        \begin{equation}
            W_{a\rightarrow b} = \int_{r_a}^{r_b} F_r dr =
            \int_{r_a}^{r_b} \frac{1}{4\pi\epsilon_0} \frac{qq_0}{r^2}dr =
            \frac{qq_0}{4\pi\epsilon_0}\bigg(\frac{1}{r_a}-\frac{1}{r_b}\bigg)
        \end{equation}
    \item The work done by the electric force for this path depends on only the
        endpoints, so the work done on $q_0$ by $\vec{\mathbf{E}}$ depends only on $r_a$
        and $r_b$, not on the details of the path.
    \item This means that if $q_0$ returns to the initial starting point, the work done
        is $0$, this means that the force on $q_0$ is conservative
    \item Electric potential energy is then defined as:
        \begin{equation}
            U = \frac{1}{4\pi\epsilon_0} \frac{qq_0}{r}
        \end{equation}
    \item
\end{itemize}

\subsection{Electric Potential Energy with Several Point Charges}
\begin{itemize}
    \item Suppose the electric field $\vec{\mathbf{E}}$ in which the charge $q_0$ moves
        is caused by several point charges $q_1, q_2, q_3, ...$ at distances
        $r_1, r_2, ...$ from $q_0$. The total electric field at each point is the vector
        sum of the fields
        \begin{equation}
            U = \frac{q_0}{4\pi\epsilon_0}\bigg(\frac{q_1}{r_1} + \frac{q_2}{r_2} + ...  \bigg)
            = \frac{q_0}{4\pi\epsilon_0} \sum_i \frac{q_i}{r_i}
        \end{equation}
    \item For every electric field due to a static charge distribution, the force exerted
        by that field is conservative
\end{itemize}

\section[23.2, Electric Potential]{Electric Potential}
\begin{itemize}
    \item Potential $V$ at any point in the electric field as the potential energy per
        unit charge, or as the potential energy $U$ per unit charge associated with $q_0$
        \begin{equation}
            V = \frac{U}{q_0}
        \end{equation}
    \item Then we can relate this to work from $a$ to $b$
        \begin{equation}
            \frac{W_{a\rightarrow b}}{q_0} = - \frac{\Delta U}{q_0} =
            -\bigg(\frac{U_b}{q_0} - \frac{U_a}{q_0} \bigg) =
            -(V_b - V_a) = V_a - V_b
        \end{equation}
    \item This equation states that the potential (in $V$) of $a$ with respect to $b$,
        equals the work (in $J$) done by the electric force when a charge moves from $a$
        to $b$
\end{itemize}

\subsection{Calculating Electric Potential}
\begin{itemize}
    \item To find the potential V due to a single point charge, $q$, we have
        \begin{equation}
            V = \frac{1}{4\pi\epsilon_0}\frac{q}{r}
        \end{equation}
    \item Similarly, when you have a collection of point charges:
        \begin{equation}
            V = \frac{1}{4\pi\epsilon_0}\sum_i \frac{q_i}{r_i}
        \end{equation}
    \item If we have a continuous distribution of charge along a line, over a surface, or
        throught a volume, we divide the charge into elements $dq$ and then integrate:
        \begin{equation}
            V = \frac{1}{4\pi\epsilon_0}\int \frac{dq}{r}
        \end{equation}
\end{itemize}

\subsection{Finding Electric Potential from Electric Field}
\begin{itemize}
    \item When given a collection of point charges, and the electric field can be found
        easily, it is easier to determine $V$ from $\vec{\mathbf{E}}$. The force
        $\vec{\mathbf{F}}$ on the test charge $q_0$ can be written as
        $\vec{\mathbf{F}} = q_0 \vec{\mathbf{E}}$ so the work done by the electric force
        is:
        \begin{equation}
            W_{a\rightarrow b} = \int_{a}^{b} \vec{\mathbf{F}} \cdot d \vec{\mathbf{l}} =
            \int_{a}^{b} q_0\vec{\mathbf{E}} \cdot d \vec{\mathbf{l}}
        \end{equation}
    \item If we divide this by $q_0$ and compare the result, we find that:
        \begin{equation}
            V_a - V_b = \int_{a}^{b} \vec{\mathbf{E}} \cdot d \vec{\mathbf{l}} =
            \int_{a}^{b} E\cos\phi dl
        \end{equation}
        Where $E$ is the electric field magnitude and $\phi$ is the angle between
        $\vec{\mathbf{E}}$ and $d \vec{\mathbf{l}}$
    \item Again the value $V_a - V_b$ is independent of the path taken from $a$ to $b$,
        just as the value of $W_{a\rightarrow b}$ is independent of path
\end{itemize}
\end{document}
