\documentclass[11pt, a4paper]{article}
\usepackage[letterpaper, portrait, margin=0.5in]{geometry}
\usepackage[english]{babel}  % force American English hyphenation patterns
\usepackage{amsmath,mathtools}

\usepackage{graphicx}
\usepackage{wrapfig}

% Use this to define a bigger dot
% Cred: https://tex.stackexchange.com/questions/235118/making-a-thicker-cdot-for-dot-product-that-is-thinner-than-bullet/235120
\makeatletter
\newcommand*\bigcdot{\mathpalette\bigcdot@{.5}}
\newcommand*\bigcdot@[2]{\mathbin{\vcenter{\hbox{\scalebox{#2}{$\m@th#1\bullet$}}}}}
\makeatother

\begin{document}
\title{Chapter 28 Sources of Magnetic field}
\author{Apostolos Delis}

\date{\today}
\maketitle

\tableofcontents
\section[28.1, Magnetic Field of a Moving Charge]{Magnetic Field of a Moving Charge}
\begin{itemize}
    \item The location of the moving charge at a given instant is called the
        \textbf{source point} and the point $P$ where we want to find the field the
        \textbf{field point}
    \item The magnitude of $\vec{\mathbf{B}}$ is proportional to $|q|$ and to
        $\frac{1}{r^2}$
    \item The magnetic field magnitude at point $P$ is:
        \begin{equation}
            B = \frac{\mu_0}{4\pi} \frac{|q|v\sin\phi}{r^2}
        \end{equation}
    \item The quantity $\mu_0$ is called the magnetic constant
    \item The vector for the magnetic field can be calculated using:
        \begin{equation}
            \vec{\mathbf{B}} = \frac{\mu_0}{4\pi}
            \frac{q\vec{\mathbf{v}} \times \mathbf{\hat{r}}}{r^2}
        \end{equation}
\end{itemize}

\section[28.2, Magnetic Field of a Current Element]{Magnetic Field of a Current Element}
\begin{itemize}
    \item \textbf{Principle of superposition of magnetic fields:} the totoal magnetic
        field caused by several moving charges is the vector sum of the fields caused by
        the individual charges.
    \item If there are $n$ moving charged particles per unit volume, each of charge $q$,
        the total moving charge $dQ$ in the segment is:
        \begin{equation}
            dQ = nqAdl
        \end{equation}
    \item The magnitude of the resulting field $d\vec{\mathbf{B}}$ at any point $P$ is:
        \begin{equation}
            dB = \frac{\mu_0}{4\pi}\frac{|dQ|v_d\sin\phi}{r^2} =
            \frac{\mu_0}{4\pi}\frac{n|q|v_d Adl\sin\phi}{r^2} =
            \frac{\mu_0}{4\pi}\frac{Idl\sin\phi}{r^2}
        \end{equation}
        since $n|q|v_d A$ equals the current $I$ in the element
    \item In vector form, using the unit vector $\mathbf{\hat{r}}$, we have:
        \begin{equation}
            dB = \frac{\mu_0}{4\pi}\frac{Id\vec{\mathbf{l}} \times \mathbf{\hat{r}}}{r^2}
        \end{equation}
        this equation is also known as the law of Biot and Savart
    \item To find the total magnetic field $\vec{\mathbf{B}}$, integrate over the region:
        \begin{equation}
            \vec{\mathbf{B}} = \frac{\mu_0}{4\pi} \int 
            \frac{Id\vec{\mathbf{l}} \times \mathbf{\hat{r}}}{r^2}
        \end{equation}
\end{itemize}

\section[28.3, Magnetic Field of a Straight Current-Carrying Conductor]{Magnetic Field of
    a Straight Current-Carrying Conductor}
\begin{itemize}
    \item It is useful to calculate the magnetic field produced by a straight
        current-carrying conductor, since straight wires are found in countless
        situations
    \item It can be found that the total mangitude of $\vec{\mathbf{B}}$ is 
        \begin{equation}
            B = \frac{\mu_0}{4\pi} \int_{-a}^{a} \frac{xdy}{(x^2 + y^2)^{3 / 2}}
        \end{equation}
    \item When the length $2a$ of the conductor is much greater than the distance $x$
        from the point $P$, we can consider it to be infinitely long. When
        $a\rightarrow\infty$, we have:
        \begin{equation}
            B = \frac{\mu_0 I}{2\pi x}
        \end{equation}
    \item This follows from Gauss's law of magnetism, which as stated previously, is
        defined as:
        \begin{equation}
            \oint \vec{\mathbf{B}} \bigcdot d\vec{\mathbf{A}} = 0
        \end{equation}
\end{itemize}

\section[28.4, Force Between Parallel Conductors]{Force Between Parallel Conductors}
\begin{itemize}
    \item Suppose there are two long, straight parallel conductors, seperated by a
        distance $r$, and carrying currents $I$ and $I^{\prime}$
    \item The first conductor produces a $\vec{\mathbf{B}}$ field that, at the position
        of the upper conductor, has magnitude 
        \begin{equation}
            B = \frac{\mu_0 I}{2\pi r}
        \end{equation}
    \item The force that the second field exerts on a length $L$ is
        $\vec{\mathbf{F}} = I^{\prime}\vec{\mathbf{L}} \times \vec{\mathbf{B}}$, where
        the vector $\vec{\mathbf{L}}$ is the direction of the current $I^{\prime}$ and
        has magnitude $L$
    \item Since $\vec{\mathbf{B}}$ is perpendicular to $\vec{\mathbf{L}}$, the magnitude
        of the force is
        \begin{equation}
            F = I^{\prime}IB = \frac{\mu_0II^{\prime}L}{2\pi r}
        \end{equation}
    \item The force per unit length $F / L$ in this scenario is
        \begin{equation}
            \frac{F}{L} = \frac{\mu_0II^{\prime}}{2\pi r}
        \end{equation}
\end{itemize}

\section[28.6 Ampere's Law]{Ampere's Law}
\begin{itemize}
    \item Ampere's Law provides a nice and simple way to calculate magnetic fields with
        highly symmetric current distributions
    \item Ampere's law is formulated not in terms of magnetic flux, but rather in terms
        of the line integral of $\vec{\mathbf{B}}$ around a closed path, denoted by
        \begin{equation}
            \oint \vec{\mathbf{B}} \bigcdot d\vec{\mathbf{l}}
        \end{equation}
\end{itemize}

\subsection{Ampere'S Law For a Long, Straight Conductor}
\begin{itemize}
    \item Consider the magnetic field caused by a long, straight conductor carrying a
        current $I$. The magnetic field lines are centered on the conductor. Take a line
        integral of $\vec{\mathbf{B}}$ around a circle with radius $r$.
    \item At every point in the circle, $\vec{\mathbf{B}}$ and $d\vec{\mathbf{d}}$ are
        parallel, so $\vec{\mathbf{B}} \bigcdot d\vec{\mathbf{l}} = Bdl$, because of
        this, $B$ can be taken outside of the integral
        \begin{equation}
            \oint \vec{\mathbf{B}} \bigcdot d\vec{\mathbf{l}} = 
            \oint B_{\parallel} dl = B\oint dl = 
            \frac{\mu_0 I}{2\pi r}(2\pi r) = \mu_0 I
        \end{equation}
    \item So for straight conductors, Ampere's law can be reduced to:
        \begin{equation}
            \oint \vec{\mathbf{B}} \bigcdot d\vec{\mathbf{l}} = \mu_0 I
        \end{equation}
\end{itemize}

\subsection{Ampere's Law: General Statement}
\begin{itemize}
    \item Ampere's Law can be generalized even further, the total magnetic field
        $\vec{\mathbf{B}}$ at any point on the path is the vector sum of the fields
        produced by the individual conductors. Thus, the line integral of the total
        $\vec{\mathbf{B}}$ is 
        \begin{equation}
            \oint \vec{\mathbf{B}} \bigcdot d\vec{\mathbf{l}} = \mu_0 I_{encl}
        \end{equation}
\end{itemize}
\end{document}
