\documentclass[11pt, a4paper]{article}
\usepackage[letterpaper, portrait, margin=0.5in]{geometry}
\usepackage[english]{babel}  % force American English hyphenation patterns
\usepackage{amsmath,mathtools}

\usepackage{graphicx}
\usepackage{wrapfig}

% Use this to define a bigger dot
% Cred: https://tex.stackexchange.com/questions/235118/making-a-thicker-cdot-for-dot-product-that-is-thinner-than-bullet/235120
\makeatletter
\newcommand*\bigcdot{\mathpalette\bigcdot@{.5}}
\newcommand*\bigcdot@[2]{\mathbin{\vcenter{\hbox{\scalebox{#2}{$\m@th#1\bullet$}}}}}
\makeatother

\begin{document}
\title{Chapter 30 Inductance}
\author{Apostolos Delis}

\date{\today}
\maketitle

\tableofcontents
\section[30.1, Mutual Inductance]{Mutual Inductance}
\begin{itemize}
    \item Suppose that there are 2 different wires, where a current flowing through
        current 1 produces a magnetic field $\vec{\mathbf{B}}$
    \item When $i_1$ changes, $\Phi_{B2}$ changes, this changing flux induces an emf
        $\mathcal{E}_2$ in coil 2, given by:
        \begin{equation}
            \mathcal{E}_2 = -N_2 \frac{d\Phi_{B2}}{dt}
        \end{equation}
    \item $\Phi_{B2}$ can be represented in the form of $\Phi_{B2} = M_{21}i_1$, where
        the proportionality constant $M_{21}$ is called mutual inductance of two coils,
        written as:
        \begin{equation}
            N_2\Phi_{B2} = M_{21}i_1
        \end{equation}
    \item Thus $\Phi_{B2}$ can be rewritten as:
        \begin{equation}
            \mathcal{E}_2 = -M_{21}\frac{di_1}{dt}
        \end{equation}
    \item Thus we can write the definition of mutual inductance as:
        \begin{equation}
            M_{21} = \frac{N_2\Phi_{B2}}{i_1}
        \end{equation}
    \item The mutually induced emfs can be written as:
        \begin{equation}
            \mathcal{E}_2 = -M\frac{di_1}{dt} \; \; \; \text{and} \; \; \; 
            \mathcal{E}_1 = -M\frac{di_2}{dt}
        \end{equation}
\end{itemize}

\end{document}
